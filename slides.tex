\documentclass{beamer}
\usepackage[latin1]{inputenc}
\usepackage{listings}
\usepackage{multicol}
\usetheme{Berlin}
\setbeamercolor{palette secondary}{use=seahorse}
\usecolortheme{seahorse}
\title[How to be a Vim Ninja]{How to be a Vim Ninja}
\author{\hspace{12pt}Robert Bittle\hspace{12pt}\\\hspace{12pt}robert.bittle@dominionenterprises.com\hspace{12pt}\\\hspace{12pt}github.com/guywithnose\hspace{12pt}}
\date{December 12, 2014}
\setbeamertemplate{itemize items}[circle]
\begin{document}
    \begin{frame}
        \titlepage
        Slides and demos at:\\
        \href{https://github.com/guywithnose/de-devcon-vim}{https://github.com/guywithnose/de-devcon-vim}\\
        \href{http://bit.ly/11po3hk}{http://bit.ly/11po3hk}
    \end{frame}


    \section{Intro}
    \begin{frame}{Who am I?}
        \begin{itemize}
            % Be quick, maybe cut this slide
            \item robert.bittle@dominionenterprises.com
            \item github.com/guywithnose
        \end{itemize}
    \end{frame}
    \subsection{Why Vim?}
    % I'm not sure if this is a good slide to keep
    % \begin{frame}{The beginners experience}
    %     \begin{itemize}
    %         \item <alert@+> Things don't work the way I expect.
    %         \item <alert@+> Vim is very complex and difficult to learn.
    %         \item <alert@+> Before you try to use vim spend some time in vimtutor
    %         % Find out where the audience is
    %         % Have you ever used vim before
    %         % Do you use it regularly?
    %         % Is it your primary editor?
    %         % Do you only use it when you are composing git commit messages?
    %     \end{itemize}
    % \end{frame}
    \begin{frame}{Why is Vim hard to learn?}
        \begin{itemize}
            \item <alert@+> Vim is extremely complex % This is a pro and a con
            \item <alert@+> Vim is full of features that are not imediately discoverable
            \item <alert@+> Vim is for experienced developers not beginners.
            % It takes time to learn the complexity, but over time you learn the commands and eventually it is all muscle memory.
        \end{itemize}
    \end{frame}
    \begin{frame}{Why is Vim worth the steep learning curve?}
        \begin{itemize}
            \item <alert@+> Vim is designed to maximize your productivity potential.
            \item <alert@+> Your vim skills can grow slowly but consistently over time
                % Vim does a little bit to get you started, but it really can
                % be a pain to beginners.  Learning vim is often done through
                % community.  The best reason to learn vim is because you've
                % seen someone you know use it properly.  One of the best
                % motivations to learn vim is know how well you saw someone
                % else use it.  It may take a while.
        \end{itemize}
    \end{frame}
    % \subsection{The power of Vim}
    % Since I want to aim this talk at intermediate vim users this might be unnecessary.  It may be worth leaving if we bring in the :move command
    % \defverbatim[colored]\lstLinesFlipped{
    %     \begin{lstlisting}
% Second line.
% First line.
    %     \end{lstlisting}
    % }
    % \defverbatim[colored]\lstArrow{
    %     \begin{lstlisting}
% -------->
    %     \end{lstlisting}
    % }
    % \defverbatim[colored]\lstLinesCorrect{
    %     \begin{lstlisting}
% First line.
% Second line.
    %     \end{lstlisting}
    % }
    % \begin{frame}{Moving text}{What we want to do}
    %     \begin{multicols}{3}
    %         \lstLinesFlipped
    %         \columnbreak
    %         \pause
    %         \lstArrow
    %         \columnbreak
    %         \lstLinesCorrect
    %     \end{multicols}
    % \end{frame}
    % \begin{frame}{Moving text}{How we can do it}
    %     \begin{itemize}
    %         \item <alert@+> Standard text editor
    %         \begin{itemize}
    %             % Using mouse
    %             \item <alert@+> Highlight with mouse and drag
    %             \item <alert@+> Highlight with mouse, Ctrl-x, Down, Ctrl-v
    %             \item <alert@+> Home, Shift-Down, Ctrl-x, Down, Ctrl-v (8 keystrokes)
    %         \end{itemize}
    %         \item <alert@+> Vim
    %         \begin{itemize}
    %             \item <alert@+> dd(Cut a line)p(Paste) (3 keystrokes)
    %         \end{itemize}
    %    \end{itemize}
    % \end{frame}
    \section{How to learn Vim}
    % I only have an hour today, so I can only scratch the surface of vim's
    % capabilities and lets be honest if I only show you a few things that I
    % think are cool you might remember 10% of this. Instead I want to point
    % you in the right direction and teach you how to learn vim because vim is
    % best learned over time.  The only way I know how to master vim is by
    % learning a small number of things at a time over a long period of time.
    \subsection{}
    % \begin{frame}{Learn the basic commands}
    %     \begin{itemize}
    %         \item <alert@+> vimtutor
    %         \item <alert@+> Vim documentation
    %     \end{itemize}
    % \end{frame}
    \begin{frame}{vimtutor}
        \begin{itemize}
            \item <alert@+> ... should be the first half hour you spend in vim.
            \item <alert@+> ... primes you with the base functionality, but only enough to get you started.
            \item <alert@+> ... most importantly shows you where to find the manual.
        \end{itemize}
    \end{frame}
    \begin{frame}{Vim documentation}
        \begin{itemize}
            \item <alert@+> ... is very exhaustive.
            \item <alert@+> ... is not intended to be read completely.  % 6MB of text
            \item <alert@+> ... is designed to be very easily searchable.
                % This is a great place to demonstrate where to find vim
                % documentation, how to search it and how to navigate it.
                % Also mention that if vim doesn't do what you are looking for
                % natively, you might be able to find a plugin for it.
        \end{itemize}
    \end{frame}
    \begin{frame}{Maintain a cheatsheet}
        \begin{itemize}
            \item <alert@+> Decide on a few (no more than 10) things you want to learn.
            \item <alert@+> Write them down where you can see them while you are working in vim.
            \item <alert@+> Force yourself to use them in real world scenarios.
                % If you do something and then realize you could have used what
                % you are learning, undo and use the new technique.  If you
                % never run into a real world scenario to use a cool feature of
                % vim, then erase it from your list and don't waste your time
                % learning it.
        \end{itemize}
    \end{frame}
    \section{Macros}
    \subsection{}
    % To really maximize the potential of vim, you need to learn to recognize
    % when you are frequently doing the same mechanical operations.  Macros
    % allow you to record these operations and then play them back.  It is
    % clear to see how this would save you keystrokes, but recording the macro
    % in a repeatable way can be tricky.
    \begin{frame}{Basic commands}
        \begin{itemize}
            \item q[0-9a-z] - start recording
            \item Do something that is repeatable
            \item q - stop recording
            \item @[0-9a-z]
            \item :help complex-repeat
        \end{itemize}
    \end{frame}
    \begin{frame}{Tips}
        \begin{itemize}
            \item Try to start the macro with an operation that will fail after the last iteration.
            \item If the macro is applied linewise to a block of consecutive lines use visual block mode.
            \item Try to avoid absolute motions.
            \item Use search motions, word motions, mark motions, etc.
        \end{itemize}
        % This definitely calls for a demo.  Create a file just for this
        % demonstration with the macros spelled out.  Maybe even include a
        % naive example and show why it doesn't work. Maybe wrap a bunch of
        % lines in an html tag.
    \end{frame}
    \section{Level up your .vimrc}
    \subsection{}
    \begin{frame}{What is a vimrc file?}
        \begin{itemize}
            \item Your vimrc file is where you store your custom vim configuration.
            \item Vim loads this information every time that it starts up.
                % So be careful not to add a lot of bloated plugins
            \item As you become a more mature vim user, your vimrc will grow to include all the special sauce that makes vim work the way you like it.
        \end{itemize}
    \end{frame}
    \begin{frame}{Mappings}
        \begin{itemize}
            \item Mappings are an easy way to bind macros to a key or combination of keys.
                % Talk about leader commands
            \item Over time as you find ways to automate common operations with macros you can save them as mappings in you vimrc file.
                % Demonstrate how to paste a recorded macro as a mapping
        \end{itemize}
    \end{frame}
    \section{Advanced techniques}
    \subsection{}
    \begin{frame}{Motions and Actions}
    \end{frame}
    \begin{frame}{Dot Command}
    \end{frame}
    % A word about plugins:
    % Get to know the base functionality first.  Many plugins extend or change
    % current functionality and add or overwrite key commands.  Make sure you
    % know what you are giving up for a plugin.  It may be better to use built
    % in functionality.  Keep in mind that every plugin you add can potentially
    % slow down vim's load time.  With that said, here a few of my favorites.
    \section{Plugins}
    \subsection{}
    \begin{frame}{My Favorite Plugins}
        \begin{itemize}
            \item Editing tasks
                \begin{itemize}
                    \item Surround
                    \item Ultisnips
                \end{itemize}
            \item Folding (Usually language specific)
            \item Interface
                \begin{itemize}
                    \item fugitive
                    \item CtrlP
                \end{itemize}
        \end{itemize}
    \end{frame}
    \begin{frame}{How to install plugins}{Pathogen}
        \begin{itemize}
            \item Add pathogen.vim to \textasciitilde/.vim/autoload
            \item Add this line to your .vimrc\\
                \quad execute pathogen\#infect()
            \item Plugins get loaded from \textasciitilde/.vim/bundle
        \end{itemize}
    \end{frame}
    \begin{frame}{How to install plugins}{Git}
        Once you have Pathogen installed
        \begin{itemize}
            \item Copy plugins to your bundle directory or
            \item if the plugin is in git you can check it out to a directory in bundle or
            \item take advantage of git submodules
                % This is really useful if you like to keep your settings
                % synchronized across two systems.
        \end{itemize}
    \end{frame}
    % Before introducing this, demonstrate what a mechanical operation is by
    % showing the motivation behind split-join
    \subsection{Crafting your own plugins}
    \begin{frame}{Directory structure}
    \end{frame}
\end{document}
